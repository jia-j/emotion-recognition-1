\documentclass[11pt]{article}
\usepackage[utf8]{inputenc}
\usepackage{amsmath,amsthm,amsfonts,amssymb,amscd}
\usepackage{multirow,booktabs}
\usepackage[table]{xcolor}
\usepackage{fullpage}
\usepackage{lastpage}
\usepackage{enumitem}
\usepackage{fancyhdr}
\usepackage{mathrsfs}
\usepackage{wrapfig}
\usepackage{setspace}
\usepackage{calc}
\usepackage{multicol}
\usepackage{cancel}
\usepackage[retainorgcmds]{IEEEtrantools}
\usepackage[margin=3cm]{geometry}
\usepackage{amsmath}
\newlength{\tabcont}
\setlength{\parindent}{0.0in}
\setlength{\parskip}{0.05in}
\usepackage{empheq}
\usepackage{framed}
\usepackage[most]{tcolorbox}
\usepackage{xcolor}
\colorlet{shadecolor}{orange!15}
\parindent 0in
\parskip 12pt
\geometry{margin=1in, headsep=0.25in}
\theoremstyle{definition}
\newtheorem{defn}{Definition}
\newtheorem{reg}{Rule}
\newtheorem{exer}{Exercise}
\newtheorem{note}{Note}

\title{\textbf{Emotion recognition using physiological signals}}
\author{
  Russel Shawn Dsouza\\
  171EC143
  \and
  Sathvik Prabhu\\
  171EC146
}
\date{today}

\begin{document}
  \maketitle

  \section{Introduction}
    Emotions, which affect both human physiological and psychological status, play a very important role in human life.\textsuperscript{[citation needed]}
    Positive emotions help improve human health and work efficiency, while negative emotions may cause health problems.\textsuperscript{[citation needed]}
    Long term accumulations of negative emotions are predisposing factors for depression, which might lead to suicide in the worst cases.\textsuperscript{[citation needed]}
    Compared to the mood which is a conscious state of mind or predominant emotion in a time, the emotion often refers to a mental state that arises spontaneously rather than through conscious effort and is often accompanied by physical and physiological changes that are relevant to the human organs and tissues such as brain, heart, skin, blood flow, muscle, facial expressions, voice, etc.\textsuperscript{[citation needed]}
    Due to the complexity of mutual interaction of physiology and psychology in emotions, recognizing human emotions precisely and timely is still limited to our knowledge and remains the target of relevant scientific research and industry, although a large number of efforts have been made by researchers in different interdisciplinary fields.

    Emotion recognition has been applied in many areas such as safe driving \textsuperscript{[citation needed]}, health care especially mental health monitoring \textsuperscript{[citation needed]}, social security \textsuperscript{[citation needed]}, and so on. In general, emotion recognition methods could be classified into two major categories. One is using human physical signals such as facial expression \textsuperscript{[citation needed]}, speech \textsuperscript{[citation needed]}, gesture, posture, etc., which has the advantage of easy collection and have been studied for years. However, the reliability can’t be guaranteed, as it’s relatively easy for people to control the physical signals like facial expression or speech to hide their real emotions especially during social communications.\textsuperscript{[citation needed]}

    Diagnosis of psychiatric diseases is currently accomplished with questionnaires filled in by the subjects, usually together with a specialist. 
    Such questionnaires are often based on standard, formal scales, where questions range from the ability to cope with household activities, through social interactions, agitation, level of activity, to quality of sleep. 
    In this paper, we propose a context aware framework to support semi-automation of the diagnosis of such diseases.

    Post Traumatic Stress Disorder (PTSD), depression and suicide are major psychiatric problem in both military and civilian situation.\textsuperscript{[citation needed]}
    These mental diseases are closely related to emotion change.\textsuperscript{[citation needed]}
    
    Additionally in rehabilitation applications, guiding patients through their rehabilitation training while adapting to the patients emotional state, would be highly motivating and might lead to a faster recovery.\textsuperscript{[citation needed]}

  \section{Motivation}

  \section{Survey of State of the Art}
    
  \section{Features we wish to implement}

  \section{Implementation Details}

  \section{Project Timeline}
  
\end{document}

